\documentclass[a4paper,11pt]{jreport}


%%
% このファイルは、筑波大学大学院システム情報工学研究科の
% 学位論文本体のサンプルです。
% このファイルを書き換えて、この例と同じような書式の論文本体を
% LaTeXを使って作成することができます。
% 
% PC環境や、LaTeX環境の設定によっては漢字コードや改行コードを
% 変更する必要があります。
%%

%%【PostScript, JPEG, PNG等の画像の貼り込み】
%% 利用するパッケージを選んでコメントアウトしてください。
%\usepackage{graphicx} % for \includegraphics[width=3cm]{sample.eps}
\usepackage[dvipdfmx]{graphicx}

\usepackage{svg}
%\usepackage{epsfig} % for \psfig{file=sample.eps,width=3cm}
%\usepackage{epsf} % for \epsfile{file=sample.eps,scale=0.6}
%\usepackage{epsbox} % for \epsfile{file=sample.eps,scale=0.6}
\usepackage{minted}

%% dvipdfm を使う場合(dvi->pdfを直接生成する場合)
%\usepackage[dvipdfm]{color,graphicx}
%% dvipdfm を使ってPDFの「しおり」を付ける場合
%\usepackage[dvipdfm,bookmarks=true,bookmarksnumbered=true,bookmarkstype=toc]{hyperref}
%% 参考:dvipdfm 日本語版
%% http://hamilcar.phys.kyushu-u.ac.jp/~hirata/dvipdfm/

\usepackage{times} % use Times Font instead of Computer Modern

\setcounter{tocdepth}{3}
\setcounter{page}{-1}

\setlength{\oddsidemargin}{0.1in}
\setlength{\evensidemargin}{0.1in} 
\setlength{\topmargin}{0in}
\setlength{\textwidth}{6in} 
%\setlength{\textheight}{10.1in}
\setlength{\parskip}{0em}
\setlength{\topsep}{0em}

%\newcommand{\zu}[1]{{\gt \bf 図\ref{#1}}}



%% ハイパーリンク用パッケージ
%%\usepackage{blindtext}
%%\usepackage{hyperref}
\usepackage[dvipdfmx]{hyperref}
\usepackage{pxjahyper}


%% タイトル生成用パッケージ(重要)
\usepackage{sie-jp-sjis}
\usepackage{docmute}

%% タイトル
%% 【注意】タイトルの最後に\\ を入れるとエラーになります
\title{HDF5のVirtual Object Layer\\を用いたI/Oの高速化}
%% 著者
\author{木下 嵩裕}
%% 学位 (2012/11 追加)
\degree{修士(工学)}
%% 指導教員
\advisor{建部 修見}


%% 専攻名 と 年月
%% 年月は必要に応じて書き替えてください。
\majorfield{hogehoge}\programfield{情報理工} \yearandmonth{2024年 3月}

\begin{document}
\maketitle
\thispagestyle{empty}
\newpage

\thispagestyle{empty}
\vspace*{20pt plus 1fil}
\parindent=1zw
\noindent
%%
%% 論文の概要(Abstract)
%%
\begin{center}
{\bf 概要}
\vspace{5mm}
\end{center}

高性能計算においては,計算機の性能向上に伴い,I/Oの性能がボトルネックとなっている.

%%%%%
\par
\vspace{0pt plus 1fil}
\newpage

\pagenumbering{roman} % I, II, III, IV 
\tableofcontents
\listoffigures
%\listoftables

\pagebreak \setcounter{page}{1}
\pagenumbering{arabic} % 1,2,3




\chapter{はじめに}

はじめに

\documentclass[a4paper,11pt]{jsreport}

%%
% このファイルは、筑波大学大学院システム情報工学研究科の
% 学位論文本体のサンプルです。
% このファイルを書き換えて、この例と同じような書式の論文本体を
% LaTeXを使って作成することができます。
% 
% PC環境や、LaTeX環境の設定によっては漢字コードや改行コードを
% 変更する必要があります。
%%

%%【PostScript, JPEG, PNG等の画像の貼り込み】
%% 利用するパッケージを選んでコメントアウトしてください。
%\usepackage{graphicx} % for \includegraphics[width=3cm]{sample.eps}
\usepackage[dvipdfmx]{graphicx}

\usepackage{svg}
%\usepackage{epsfig} % for \psfig{file=sample.eps,width=3cm}
%\usepackage{epsf} % for \epsfile{file=sample.eps,scale=0.6}
%\usepackage{epsbox} % for \epsfile{file=sample.eps,scale=0.6}
\usepackage{minted}

%% dvipdfm を使う場合(dvi->pdfを直接生成する場合)
%\usepackage[dvipdfm]{color,graphicx}
%% dvipdfm を使ってPDFの「しおり」を付ける場合
%\usepackage[dvipdfm,bookmarks=true,bookmarksnumbered=true,bookmarkstype=toc]{hyperref}
%% 参考:dvipdfm 日本語版
%% http://hamilcar.phys.kyushu-u.ac.jp/~hirata/dvipdfm/

\usepackage{times} % use Times Font instead of Computer Modern

\setcounter{tocdepth}{3}
\setcounter{page}{-1}

\setlength{\oddsidemargin}{0.1in}
\setlength{\evensidemargin}{0.1in} 
\setlength{\topmargin}{0in}
\setlength{\textwidth}{6in} 
%\setlength{\textheight}{10.1in}
\setlength{\parskip}{0em}
\setlength{\topsep}{0em}

%\newcommand{\zu}[1]{{\gt \bf 図\ref{#1}}}



%% ハイパーリンク用パッケージ
%%\usepackage{blindtext}
%%\usepackage{hyperref}
\usepackage[dvipdfmx]{hyperref}
\usepackage{pxjahyper}


\begin{document}

\subsection{h5bench write}

素粒子物理シミュレーションのI/Oパターンに基づいたVPIC-IOと,
ビックデータのクラスタリングアルゴリズムのI/Oパターンに基づいたBDCATS-IO\cite{BDCATS}カーネルが含まれている.

\begin{table}[!ht]
  \caption{h5bench writeのオプション}
  \centering

  \begin{tabular}{|l|l|}
  \hline
      MEM\_PATTERN & CONTIG, INTERLEAVED, STRIDED \\ \hline
      FILE\_PATTERN & CONTIG, STRIDED \\ \hline
      TIMESTEPS & 繰り返しの回数 \\ \hline
      EMULATED\_COMPUTE\_TIME\_PER\_TIMESTEP & コンピュート時間(繰り返しの間の時間) \\ \hline
      NUM\_DIMS & 次元数 \\ \hline
      DIM\_1 & 1次元のサイズ \\ \hline
      DIM\_2 & 2次元のサイズ \\ \hline
      DIM\_3 & 3次元のサイズ \\ \hline
  \end{tabular}
\end{table}

MEM\_PATTERNについて,CONTIGは基本データ型のint, float, doubleなどの配列を想定し,
INTERLEAVEDは,array of structureを想定し,
STRIDEDは,一定のstride幅で区切られた基本データの方のはは

\end{document}




\chapter*{謝辞}
\addcontentsline{toc}{chapter}{\numberline{}謝辞}

本研究を進めるにあたって、指導教員である筑波大学計算科学研究センター教授 建部修見先生には多大なご指導を頂きました。
また、国内外で様々な機会を与えていただきました。
心より感謝申し上げます。

TODO 副査の先生方の名前を書く \\
大変お忙しい中で本論文の副査を引き受けてくださりました。
感謝申し上げます。

研究室の卒業・修了生の方々、同期・後輩の皆様、また研究室の秘書を務められている桑野洋子様には、研究の議論のみならず様々な場面で大変お世話になりました。
感謝申し上げます。

最後に研究生活を支えてくれた友人と家族に深く感謝します。

\newpage

%\chapter*{参考文献}
\addcontentsline{toc}{chapter}{\numberline{}参考文献}
\renewcommand{\bibname}{参考文献}

%% 参考文献に jbibtex を使う場合
\bibliographystyle{junsrt}
\bibliography{references}
%% [compile] jbibtex sample; platex sample; platex sample;


\end{document}
