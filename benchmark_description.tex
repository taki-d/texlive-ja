\documentclass[a4paper,11pt]{jsreport}

%%
% このファイルは、筑波大学大学院システム情報工学研究科の
% 学位論文本体のサンプルです。
% このファイルを書き換えて、この例と同じような書式の論文本体を
% LaTeXを使って作成することができます。
% 
% PC環境や、LaTeX環境の設定によっては漢字コードや改行コードを
% 変更する必要があります。
%%

%%【PostScript, JPEG, PNG等の画像の貼り込み】
%% 利用するパッケージを選んでコメントアウトしてください。
%\usepackage{graphicx} % for \includegraphics[width=3cm]{sample.eps}
\usepackage[dvipdfmx]{graphicx}

\usepackage{svg}
%\usepackage{epsfig} % for \psfig{file=sample.eps,width=3cm}
%\usepackage{epsf} % for \epsfile{file=sample.eps,scale=0.6}
%\usepackage{epsbox} % for \epsfile{file=sample.eps,scale=0.6}
\usepackage{minted}

%% dvipdfm を使う場合(dvi->pdfを直接生成する場合)
%\usepackage[dvipdfm]{color,graphicx}
%% dvipdfm を使ってPDFの「しおり」を付ける場合
%\usepackage[dvipdfm,bookmarks=true,bookmarksnumbered=true,bookmarkstype=toc]{hyperref}
%% 参考:dvipdfm 日本語版
%% http://hamilcar.phys.kyushu-u.ac.jp/~hirata/dvipdfm/

\usepackage{times} % use Times Font instead of Computer Modern

\setcounter{tocdepth}{3}
\setcounter{page}{-1}

\setlength{\oddsidemargin}{0.1in}
\setlength{\evensidemargin}{0.1in} 
\setlength{\topmargin}{0in}
\setlength{\textwidth}{6in} 
%\setlength{\textheight}{10.1in}
\setlength{\parskip}{0em}
\setlength{\topsep}{0em}

%\newcommand{\zu}[1]{{\gt \bf 図\ref{#1}}}



%% ハイパーリンク用パッケージ
%%\usepackage{blindtext}
%%\usepackage{hyperref}
\usepackage[dvipdfmx,setpagesize=false]{hyperref}
\usepackage{pxjahyper}


\begin{document}

\subsection{h5bench write}

素粒子物理シミュレーションのI/Oパターンに基づいたVPIC-IOと,
ビックデータのクラスタリングアルゴリズムのI/Oパターンに基づいたBDCATS-IO\cite{BDCATS}カーネルが含まれている.

\begin{table}[!ht]
  \caption{h5bench writeのオプション}
  \centering

  \begin{tabular}{|l|l|}
  \hline
      MEM\_PATTERN & CONTIG, INTERLEAVED, STRIDED \\ \hline
      FILE\_PATTERN & CONTIG, STRIDED \\ \hline
      TIMESTEPS & 繰り返しの回数 \\ \hline
      EMULATED\_COMPUTE\_TIME\_PER\_TIMESTEP & コンピュート時間(繰り返しの間の時間) \\ \hline
      NUM\_DIMS & 次元数 \\ \hline
      DIM\_1 & 1次元のサイズ \\ \hline
      DIM\_2 & 2次元のサイズ \\ \hline
      DIM\_3 & 3次元のサイズ \\ \hline
  \end{tabular}
\end{table}

MEM_PATTERNについて,CONTIGは基本データ型のint, float, doubleなどの配列を想定し,
INTERLEAVEDは,array of structureを想定し,
STRIDEDは,一定のstride幅で区切られた基本データの方のはは

\end{document}

